%%%%%%%%%%%%%%%%%%%%%%%%%%%%%%%%%%%%%%%%%
% Beamer Presentation
% LaTeX Template
% Version 1.0 (10/11/12)
%
% This template has been downloaded from:
% http://www.LaTeXTemplates.com
%
% License:
% CC BY-NC-SA 3.0 (http://creativecommons.org/licenses/by-nc-sa/3.0/)
%
%%%%%%%%%%%%%%%%%%%%%%%%%%%%%%%%%%%%%%%%%

%----------------------------------------------------------------------------------------
%	PACKAGES AND THEMES
%----------------------------------------------------------------------------------------

\documentclass{beamer}

\mode<presentation> {

% The Beamer class comes with a number of default slide themes
% which change the colors and layouts of slides. Below this is a list
% of all the themes, uncomment each in turn to see what they look like.

%\usetheme{default}
%\usetheme{AnnArbor}
%\usetheme{Antibes}
%\usetheme{Bergen}
%\usetheme{Berkeley}
%\usetheme{Berlin}
%\usetheme{Boadilla}
%\usetheme{CambridgeUS}
%\usetheme{Copenhagen}
%\usetheme{Darmstadt}
\usetheme{Dresden}
%\usetheme{Frankfurt}
%\usetheme{Goettingen}
%\usetheme{Hannover}
%\usetheme{Ilmenau}
%\usetheme{JuanLesPins}
%\usetheme{Luebeck}
%\usetheme{Madrid}
%\usetheme{Malmoe}
%\usetheme{Marburg}
%\usetheme{Montpellier}
%\usetheme{PaloAlto}
%\usetheme{Pittsburgh}
%\usetheme{Rochester}
%\usetheme{Singapore}
%\usetheme{Szeged}
%\usetheme{Warsaw}

% As well as themes, the Beamer class has a number of color themes
% for any slide theme. Uncomment each of these in turn to see how it
% changes the colors of your current slide theme.

%\usecolortheme{albatross}
%\usecolortheme{beaver}
%\usecolortheme{beetle}
%\usecolortheme{crane}
%\usecolortheme{dolphin}
%\usecolortheme{dove}
%\usecolortheme{fly}
%\usecolortheme{lily}
%\usecolortheme{orchid}
%\usecolortheme{rose}
%\usecolortheme{seagull}
%\usecolortheme{seahorse}
%\usecolortheme{whale}
%\usecolortheme{wolverine}

%\setbeamertemplate{footline} % To remove the footer line in all slides uncomment this line
%\setbeamertemplate{footline}[page number] % To replace the footer line in all slides with a simple slide count uncomment this line

%\setbeamertemplate{navigation symbols}{} % To remove the navigation symbols from the bottom of all slides uncomment this line
}

\usepackage{graphicx} % Allows including images
\usepackage{booktabs} % Allows the use of \toprule, \midrule and \bottomrule in 
%tables
\usepackage{listings}
\usepackage{hyperref}
\usepackage{subfig}
\usepackage[export]{adjustbox}
\usepackage{wrapfig}

%----------------------------------------------------------------------------------------
%	TITLE PAGE
%----------------------------------------------------------------------------------------

\title[Lecture 5]{Advanced R Programming - Lecture 5} % The short title 
%appears at the bottom of every slide, the full title is only on the title page

\author{Leif Jonsson} % Your name
\institute[STIMA LiU] % Your institution as it will appear on the bottom of 
%every 
%slide, may be shorthand to save space
{
Link\"{o}ping University \\ % Your institution for the title page
\medskip
\textit{leif.jonsson@ericsson.com\\leif.r.jonsson@liu.se} % Your email address
}
\date{\today} % Date, can be changed to a custom date

\addtobeamertemplate{navigation symbols}{}{ \hspace{1em}    
\usebeamerfont{footline}%
	\insertframenumber / \inserttotalframenumber }

\begin{document}

\begin{frame}
\titlepage % Print the title page as the first slide
\end{frame}

\begin{frame}
\frametitle{Today} % Table of contents slide, comment this block out to remove 
%it
\tableofcontents % Throughout your presentation, if you choose to use \section{} and \subsection{} commands, these will automatically be printed on this slide as an overview of your presentation
\end{frame}

%----------------------------------------------------------------------------------------
%	PRESENTATION SLIDES
%----------------------------------------------------------------------------------------

\begin{frame}
	\Huge{\centerline{Questions since last time?}}
\end{frame}

%------------------------------------------------
\section{Input and output}
%------------------------------------------------

\begin{frame}
	\frametitle{Input and output}
	\begin{center}
	\begin{figure}
		\includegraphics[scale=0.4,valign=m]{figures/io}
		\label{fig:io}
	\end{figure}
	Format, localization
	\end{center}
\end{frame}

\begin{frame}
	\frametitle{Input and output}
	\begin{center}
	\begin{figure}
		\includegraphics[scale=0.4,valign=m]{figures/io}
		\label{fig:io}
	\end{figure}
	Format, localization and encoding...... hell! \\~\\
	\href{http://www.joelonsoftware.com/articles/Unicode.html}{The Absolute 
	Minimum Every Software Developer Absolutely, Positively Must Know About 
	Unicode and Character Sets (No Excuses!)}
\end{center}
\end{frame}

\begin{frame}
	\frametitle{"Formats"}
	\begin{center}
		\begin{figure}
			\includegraphics[scale=0.4,valign=m]{figures/io-formats}
			\label{fig:io}
		\end{figure}
	\end{center}
\end{frame}

\begin{frame}
	\frametitle{Localization}
	\begin{table}[t]
		\begin{center}
			\begin{tabular}{ c c }
				\includegraphics[scale=0.5]{figures/house} & 
				\includegraphics[scale=0.5]{figures/globe} \\
				own Computer & Cloud Storage \\
				local network & web pages \\
				local database & web scraping \\
				& web APIs \\
				& remote database \\
			\end{tabular}
		\end{center}
		\caption{Local - Remote}
	\end{table}
\end{frame}


%------------------------------------------------
\section{Basic I/O}
%------------------------------------------------

\defverbatim[colored]\lstBasicIOI{
	\begin{lstlisting}[language=R,basicstyle=\ttfamily,keywordstyle=\color{black}]
	# Input simple data
	read.table()
	read.csv()
	read.csv2()
	
	load()
	
	
	# Output simple data
	write.table()
	write.csv()
	write.csv2()
	
	save()
	\end{lstlisting}
}

\begin{frame}
	\frametitle{Files on your computer}
	\lstBasicIOI
\end{frame}

\begin{frame}
	\frametitle{More complex formats}
	\begin{table}[t]
		\begin{center}
			\begin{tabular}{ l l }
				\textbf{software/data} & 
				\textbf{package} \\
				Excel & XLConnect \\
				SAS, SPSS, STATA, ... & foreign \\
				XML & xml \\
				JSON (GeoJSON) & rjsonio, RJSON \\
				Documents & tm \\
				Maps & sp \\
				Images & raster \\
			\end{tabular}
		\end{center}
		\caption{Format - R package}
	\end{table}
\end{frame}


%------------------------------------------------
\section{Cloud storage}
%------------------------------------------------

\begin{frame}
	\frametitle{Cloud storage}
	\begin{table}[t]
		\begin{center}
			\begin{tabular}{ c c c c c }
				\includegraphics[scale=0.5]{figures/cloud} & 
				\includegraphics[scale=0.6]{figures/fat-right-arrow} & 
				\includegraphics[scale=0.5]{figures/paren-house} &
				\includegraphics[scale=0.6]{figures/fat-right-arrow} & 
				\includegraphics[scale=0.5]{figures/r-logo}	\\
				\includegraphics[scale=0.5]{figures/r-logo} & 
				\includegraphics[scale=0.6]{figures/fat-right-arrow} & 
				\includegraphics[scale=0.5]{figures/paren-house} &
				\includegraphics[scale=0.6]{figures/fat-right-arrow} & 
				\includegraphics[scale=0.5]{figures/cloud}	\\
			\end{tabular}
		\end{center}
		\caption{Local - Remote}
	\end{table}
\end{frame}

\begin{frame}
	\frametitle{Why?}
	\begin{center}
		Robust  \\~\\
		
		Backups  \\~\\
		
		Cloud computing  \\~\\
		
		... but how about safety? \\~\\
		
		... and can be tricky in the beginning \\~\\
	\end{center}
\end{frame}

\begin{frame}
	\frametitle{Localization}
	\begin{table}[t]
		\begin{center}
			\begin{tabular}{ l c c }
				Arbitrary data & & Structured data \\
				\includegraphics[scale=0.5]{figures/amazon-logo} & & 
				\includegraphics[scale=0.5]{figures/globe} \\
				& & \\
				\multicolumn{1}{r}{\includegraphics[scale=0.4]{figures/dropbox-logo}}
				 & 
				\includegraphics[scale=0.4]{figures/github-logo} &  \\
			\end{tabular}
		\end{center}
	\end{table}
\end{frame}

\begin{frame}
	\frametitle{API Packages}
	\begin{table}[t]
		\begin{center}
			\begin{tabular}{ l l }
				\textbf{Remote} & 
				\textbf{package} \\
				\hline
				General & downloader \\
				GitHub & repmis, downloader \\
				Dropbox & rdrop \\
				Amazon & RAmazonS3 \\
				Google Docs & googlesheets \\
			\end{tabular}
		\end{center}
	\end{table}
\end{frame}

%------------------------------------------------
\section{web API:s}
%------------------------------------------------

\begin{frame}
	\frametitle{web API:s}
	\begin{center}
		application program interface using http  \\~\\
		"contract to 'get data' online"  \\~\\
		more and more common  \\~\\
		\textbf{examples:}  \\~\\
		github \\~\\
		Riksdagen \\~\\
		Statistics Sweden \\~\\
	\end{center}
\end{frame}

\begin{frame}
	\frametitle{RESTful}
	\begin{center}
		\textbf{Basic principles:} \\~\\
		Data is returned (JSON / XML) \\~\\
		Each specific data has its own URI \\~\\
		Communication is based on HTTP verbs \\~\\
	\end{center}
\end{frame}

\begin{frame}
	\frametitle{Hypertext Transfer Protocol (http)}
	\begin{center}
		\begin{figure}
			\includegraphics[scale=0.4,valign=m]{figures/http1-url-structure}
			\label{fig:io}
		\end{figure}
	\end{center}
\end{frame}

\begin{frame}
	\frametitle{Hypertext Transfer Protocol (http)}
	\begin{center}
		\begin{figure}
			\includegraphics[scale=0.4,valign=m]{figures/http1-req-res-details}
			\label{fig:io}
		\end{figure}
	\end{center}
\end{frame}

\begin{frame}
	\frametitle{Verbs}
	\begin{table}[t]
		\begin{center}
			\begin{tabular}{ l l }
				\textbf{Verb} & 
				\textbf{Description} \\
				\hline
				GET & Get "data" from server. \\
				POST & Post "data" to server (to get something) \\
				PUT & Update "data" on server \\
				DELETE & Delete resource on server \\
			\end{tabular}
		\end{center}
	\end{table}
\end{frame}

\begin{frame}
	\frametitle{Status codes}
	\begin{table}[t]
		\begin{center}
			\begin{tabular}{ l l }
				\textbf{Code} & 
				\textbf{Description} \\
				\hline
				1XX & Information from server \\
				2XX & Yay! Gimme' data! \\
				3XX & Redirections \\
				4XX & You failed \\
				5XX & Server failed \\
			\end{tabular}
		\end{center}
	\end{table}
\end{frame}

\begin{frame}
	\frametitle{Example REST API's}
	\begin{center}
		\href{http://www.linkoping.se/open/data/Luftkvalitet/}{Link\"{o}ping 
		Luftkvalitet API}
		 \\~\\
		\href{https://developers.google.com/maps/documentation/geocoding/intro}{Google
		 Map Geocode API} \\~\\
	\end{center}
\end{frame}

\begin{frame}
	\frametitle{Common API formats}
	\begin{center}
		\textbf{JavaScript Object Notation (JSON)} \\
		Think of named lists in R \\
		R Packages: RJSONIO, rjsonlite \\~\\
		
		\textbf{Extensible Markup Language (XML)} \\
		Older format (using nodes) \\
		xpath \\
		R Packages: XML
	\end{center}
\end{frame}

\defverbatim[colored]\lstJSON{
	\begin{lstlisting}[language=PHP,basicstyle=\ttfamily\small,keywordstyle=\color{black}]
{
  "firstName": "John",
  "lastName": "Smith",
  "age": 25,
  "address": {
  	"streetAddress": "21 2nd Street",
  	"city": "New York",
  	"state": "NY",
  	"postalCode": "10021"
  },
  "phoneNumber": [
  	{ "type": "home", "number": "212 555" },
	{ "type": "fax", "number": "646 555" }
  ],
  "newSubscription": false,
  "companyName": null
}
\end{lstlisting}
}

\begin{frame}
	\frametitle{JSON}
	\lstJSON
\end{frame}


\defverbatim[colored]\lstXML{
	\begin{lstlisting}[language=xml,basicstyle=\ttfamily\tiny,keywordstyle=\color{red}]
<?xml version="1.0" encoding="utf-8"?>
<wikimedia>
<projects>
<project name="Wikipedia" launch="2001-01-05">
<editions>
<edition language="English">en.wikipedia.org</edition>
<edition language="German">de.wikipedia.org</edition>
<edition language="French">fr.wikipedia.org</edition>
<edition language="Polish">pl.wikipedia.org</edition>
<edition language="Spanish">es.wikipedia.org</edition>
</editions>
</project>
<project name="Wiktionary" launch="2002-12-12">
<editions>
<edition language="English">en.wiktionary.org</edition>
<edition language="French">fr.wiktionary.org</edition>
<edition language="Vietnamese">vi.wiktionary.org</edition>
<edition language="Turkish">tr.wiktionary.org</edition>
<edition language="Spanish">es.wiktionary.org</edition>
</editions>
</project>
</projects>
</wikimedia>
	\end{lstlisting}
}

\begin{frame}
	\frametitle{XML}
	\lstXML
\end{frame}

%------------------------------------------------
\section{web scraping}
%------------------------------------------------

\begin{frame}
	\frametitle{web scraping}
	\begin{center}
		Unstructured http(s) data \\~\\
		
		Often HTML format \\~\\
		
		Spiders / scraping / web crawlers \\~\\
		
		Basics behind search engines
	\end{center}
\end{frame}

\defverbatim[colored]\lstHTML{
	\begin{lstlisting}[language=html,basicstyle=\ttfamily\small,keywordstyle=\color{red}]
<!DOCTYPE html>
<html>
  <head>
    <title>This is a title</title>
  </head>
  <body>
    <p>Hello world!</p>
  </body>
</html>
	\end{lstlisting}
}

\begin{frame}
	\frametitle{HTML}
	\lstHTML
\end{frame}

\begin{frame}
	\frametitle{(har)rvest}
		\begin{wrapfigure}{r}{0.3\textwidth}
			\begin{center}
				\includegraphics[width=0.25\textwidth]{figures/spider_man}
			\end{center}
			\caption{Spiderman}
		\end{wrapfigure}
		\textbf{JavaScript Object Notation (JSON)} \\
		Simplify spider activity \\~\\
		Download data \\
		Parse data \\
		
		Follow links \\
		Fill out forms \\
		Store crawling history
\end{frame}

\begin{frame}
	\frametitle{Difficulties and bad spiders}
	\begin{wrapfigure}{r}{0.3\textwidth}
		\begin{center}
			\includegraphics[width=0.40\textwidth]{figures/spider_enemies}
		\end{center}
		\caption{Bad spiders}
	\end{wrapfigure}
	Scraping is fragile! \\
	Difficulties and bad spiders \\
	www.domain.se/robot.txt \\
	Politeness \\~\\
	robot traps \\
	javascript \\
	delays \\
\end{frame}

%------------------------------------------------
\section{Shiny}
%------------------------------------------------

\begin{frame}
	\frametitle{Shiny?}
	\begin{center}
		Interactive dashboards made easy \\~\\
		\includegraphics[width=0.10\textwidth]{figures/RStudio-Ball} \\~\\
		online or local \\~\\
		R as ''backend'' \\~\\
	\end{center}
\end{frame}

\begin{frame}
	\frametitle{Shiny?}
	\begin{center}
	\href{https://www.rstudio.com/products/shiny/shiny-user-showcase/}{Shiny 
		Examples}
	\end{center}
\end{frame}

\begin{frame}
	\frametitle{How it works}
	\begin{center}
		Application \\~\\
		Reactive \\~\\
		modify using HTML \\~\\
	\end{center}
			
	\texttt{MyAppName/server.R} \\
	\texttt{MyAppName/ui.R}  \\~\\
	
	\begin{center}
		server.R define working directory
	\end{center}
\end{frame}

\defverbatim[colored]\lstShiny{
	\begin{lstlisting}[language=C,basicstyle=\ttfamily\small,keywordstyle=\color{red}]
	library(shiny)
	# Examples with code
	runExample("01_hello")
	runExample("03_reactivity")
	\end{lstlisting}
}

\begin{frame}
	\frametitle{Shiny Example}
	\lstShiny
\end{frame}

\begin{frame}
	\frametitle{Publish Shiny}
	\begin{table}[t]
		\begin{center}
			\begin{tabular}{ c c }
				\includegraphics[width=0.15\textwidth]{figures/house}  & 
				\begin{tabular}[x]{@{}c@{}}locally\\zip-file in cloud \\github 
				(see 
					runGithub() ) \end{tabular} \\
			\end{tabular}
		\end{center}
	\end{table}
\end{frame}

\begin{frame}
	\frametitle{Publish Shiny}
	\begin{table}[t]
	\begin{center}
		\begin{tabular}{ c c }
			\includegraphics[width=0.15\textwidth]{figures/house}  & 
			\begin{tabular}[x]{@{}c@{}}locally\\zip-file in cloud \\github (see 
			runGithub() ) \end{tabular} \\
			\includegraphics[width=0.15\textwidth]{figures/globe}  & 
			\begin{tabular}[x]{@{}c@{}}your own server \\
				shinyapps.io \end{tabular} \\
			\end{tabular}
	\end{center}
\end{table}
\end{frame}

%------------------------------------------------
\section{Relational Databases}
%------------------------------------------------

\begin{frame}
	\frametitle{Relational Databases}
	\begin{center}
		Structured datasbase in tables \\~\\
		local or online \\~\\
		query language for I/O \\~\\
		effective for big data \\~\\
		difficult to design \\~\\
	\end{center}
\end{frame}

\begin{frame}
	\begin{center}
		\includegraphics[width=0.90\textwidth]{figures/Relational_model_concepts}
	\end{center}
\end{frame}

\begin{frame}
	\begin{center}
		\includegraphics[width=0.75\textwidth]{figures/relational_schema_with_many}
	\end{center}
\end{frame}

\begin{frame}
	\frametitle{A good database}
	\begin{center}
		Can be difficult to design \\
		No duplicates \\
		No redundancies \\
		Easy to query \\
		Easy to update \\
		"Normal forms" \\
	\end{center}
\end{frame}

\begin{frame}
	\frametitle{Using databases from R}
	\begin{table}[t]
		\begin{center}
			\begin{tabular}{ l l }
				\textbf{Database system} & 
				\textbf{R package} \\
				ODBC (Microsoft Access) & RODBC \\
				PostgreSQL & RPostgresql \\
				Oracle & ROracle \\
				MySQL & RMySql \\
				MongoDB & rmongodb \\
			\end{tabular}
		\end{center}
		\caption{Database - R package}
	\end{table}
\end{frame}

%------------------- THE END ---------------------------------------------------

\begin{frame}
\Huge{\centerline{The End... for today.}}
\Huge{\centerline{Questions?}}
\Huge{\centerline{See you next time!}}
\end{frame}

%-------------------------------------------------------------------------------

\end{document} 